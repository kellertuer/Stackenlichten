% !TEX TS-program = xelatex
% !TEX TS-options = --shell-escape

%
% A unified file to generate the sketches (PDF/Vectorgraphics) of the
%   Stackenblocken Pixel construction
%
% @kellertuer, 2017-01-01
\documentclass[a4paper]{scrartcl}
\usepackage[no-math]{fontspec}
\usepackage{xunicode,xltxtra}
\usepackage{tikz}
\usepackage{unicode-math}
\usetikzlibrary{calc,external}
\tikzexternalize
% Font
\defaultfontfeatures{Mapping=tex-text}
\setromanfont[Ligatures={Common},ItalicFont={Vollkorn-Italic},BoldFont={Vollkorn-Bold}, BoldItalicFont={Vollkorn-BoldItalic}]{Vollkorn}
%
% Styles
\tikzset{base/.style={gray,densely dotted,thin}}
\tikzset{main/.style={thick}}
\tikzset{minor/.style={dashed}}
\tikzset{holes/.style={thick,fill=yellow!20}}
\tikzset{magnets/.style={thick,fill=cyan!20}}
\tikzset{measure/.style={thin,|-|}}
\tikzset{emph/.style={fill=cyan!10!white,cyan!60!white}}
\tikzset{neo/.style={fill=cyan!20!white,cyan!60!white}}
\tikzset{every node/.style={font=\tiny}}
%
% Display lengths
\newif\ifshowLengths
\showLengthstrue
%
%
% Sizes
%
% if unsure leave them as they're because that's the production standard for now
\newcommand\wTh{0.4} %woodThickness in cm, short W in comments
\newcommand\mTh{0.4} %MagnetThickness in cm, short M in comments
\newcommand\pTh{0.4} %plexi frontplare Thickness
\newcommand\sL{8.0} %length of a side of the Triangle
\newcommand\h{8.0} % height of the side
\begin{document}
  \begin{figure}\centering
    \tikzsetnextfilename{sketch-top}
    \begin{tikzpicture}
      %emhasized elements on the top sides of the wood side elements
      \draw[emph] (-\sL/16,0) rectangle (\sL/16,\wTh);
      \draw[emph] (-5*\sL/16,0) rectangle (-3*\sL/16,\wTh);
      \draw[emph] (5*\sL/16,0) rectangle (3*\sL/16,\wTh);
      % a base line
      \draw[base] (-\sL/2,0) -- (\sL/2,0);
      % wood boundary
      \draw[main] ({- ( \sL/2 -  \wTh*cot(60) )}, \wTh) -- ({\sL/2 -  \wTh*cot(60)},\wTh);
      \draw[main] ({- ( \sL/2 -  \wTh*cot(60) )}, 0) -- ({\sL/2 -  \wTh*cot(60)},0);
      \draw[main] ({- ( \sL/2 -  \wTh*cot(60) )}, \wTh) -- ++(0,-\wTh);
      \draw[main] ({\sL/2 -  \wTh*cot(60)}, \wTh) -- ++(0,-\wTh);
      % Left side
      \begin{scope}[shift={(-\sL/4,{1/2*sqrt((1-1/4)*\sL*\sL)})},rotate=60]
        \draw[base] (-\sL/2,0) -- (\sL/2,0);
        \draw[emph] (-\sL/16,0) rectangle (\sL/16,-\wTh);
        \draw[emph] (-5*\sL/16,0) rectangle (-3*\sL/16,-\wTh);
        \draw[emph] (5*\sL/16,0) rectangle (3*\sL/16,-\wTh);
       \draw[main] ({- ( \sL/2 -  \wTh*cot(60) )},-\wTh) -- ({\sL/2 -  \wTh*cot(60)},-\wTh);
        \draw[main] ({- ( \sL/2 -  \wTh*cot(60) )}, 0) -- ({\sL/2 -  \wTh*cot(60)},0);
        \draw[main] ({- ( \sL/2 -  \wTh*cot(60) )},-\wTh) -- ++(0,\wTh);
        \draw[main] ({\sL/2 -  \wTh*cot(60)},-\wTh) -- ++(0,\wTh);
      \end{scope}
      %right side
      \begin{scope}[shift={(\sL/4,{1/2*sqrt((1-1/4)*\sL*\sL)})},rotate=-60]
        \draw[base] (-\sL/2,0) -- (\sL/2,0);
        \draw[emph] (-\sL/16,0) rectangle (\sL/16,-\wTh);
        \draw[emph] (-5*\sL/16,0) rectangle (-3*\sL/16,-\wTh);
        \draw[emph] (5*\sL/16,0) rectangle (3*\sL/16,-\wTh);
       \draw[main] ({- ( \sL/2 -  \wTh*cot(60) )},-\wTh) -- ({\sL/2 -  \wTh*cot(60)},-\wTh);
        \draw[main] ({- ( \sL/2 -  \wTh*cot(60) )}, 0) -- ({\sL/2 -  \wTh*cot(60)},0);
        \draw[main] ({- ( \sL/2 -  \wTh*cot(60) )},-\wTh) -- ++(0,\wTh);
        \draw[main] ({\sL/2 -  \wTh*cot(60)},-\wTh) -- ++(0,\wTh);
      \end{scope}
      % Neopixel in Scale
      \draw[neo] (0,{1/3*sqrt((1-1/4)*\sL*\sL)}) circle (.5cm);
      \ifshowLengths
      % Measures
        \draw[measure] ({- ( \sL/2 -  \wTh*cot(60) )},-.15) -- (-5*\sL/16,-.15)
          node[midway,below] {\pgfmathparse{10*(3*\sL/16 - \wTh*cot(60))}%
          \pgfmathprintnumber{\pgfmathresult}\,mm};
         \draw[measure] (-5*\sL/16,-.15-.25) -- (-3*\sL/16,-.15-.25) node[midway, below]{\pgfmathparse{10*\sL/8}%
                \pgfmathprintnumber{\pgfmathresult}\,mm};
         \draw[measure] (-3*\sL/16,-.15) -- (-\sL/16,-.15) node[midway, below]{\pgfmathparse{10*\sL/8}%
                \pgfmathprintnumber{\pgfmathresult}\,mm};
         \draw[measure] (-\sL/16,-.15-.25) -- (\sL/16,-.15-.25) node[midway, below]{\pgfmathparse{10*\sL/8}%
                \pgfmathprintnumber{\pgfmathresult}\,mm};
         \draw[measure] (\sL/16,-.15) -- (3*\sL/16,-.15) node[midway, below]{\pgfmathparse{10*\sL/8}%
                \pgfmathprintnumber{\pgfmathresult}\,mm};
         \draw[measure] (3*\sL/16,-.15-.25) -- (5*\sL/16,-.15-.25) node[midway, below]{\pgfmathparse{10*\sL/8}%
                \pgfmathprintnumber{\pgfmathresult}\,mm};
        \draw[measure] ({( \sL/2 -  \wTh*cot(60) )},-.15) -- (5*\sL/16,-.15)
          node[midway,below] {\pgfmathparse{10*(3*\sL/16 - \wTh*cot(60))}%
          \pgfmathprintnumber{\pgfmathresult}\,mm};
      \draw[measure] ({- ( \sL/2 -  \wTh*cot(60) )},-.8) -- ({\sL/2 -  \wTh*cot(60)},-.8)
         node[midway,below] {\pgfmathparse{10*(\sL - 2*\wTh*cot(60))}%
         \pgfmathprintnumber{\pgfmathresult}\,mm};
      \draw[measure] (-\sL/2,-1.45) -- (\sL/2,-1.45) node [midway, below]{\pgfmathparse{10*\sL}%
                \pgfmathprintnumber{\pgfmathresult}\,mm};
      \fi
    \end{tikzpicture}
    \caption{Top view with side plates}
  \end{figure}
  \begin{figure}\centering
    \tikzsetnextfilename{sketch-top-plexi}
    \begin{tikzpicture}
      \draw[base] (-\sL/2,0) -- (\sL/2,0);
      \draw[main] (-\sL/2,0) -- (-5*\sL/16,0) -- (-5*\sL/16,\wTh)
         -- (-3*\sL/16,\wTh) -- (-1.5,0) -- (-\sL/16,0) -- (-\sL/16,\wTh)
         -- (\sL/16,\wTh) -- (\sL/16,0) -- (3*\sL/16,0) -- (3*\sL/16,\wTh)
         -- (5*\sL/16,\wTh) -- (5*\sL/16,0) -- (\sL/2,0);
      \begin{scope}[shift={(-2,{2*sqrt(3)})},rotate=60] % shift left and halft trig height
      \draw[base] (-\sL/2,0) -- (\sL/2,0);
      \draw[main] (-\sL/2,0) -- (-5*\sL/16,0) -- (-5*\sL/16,-\wTh)
         -- (-3*\sL/16,-\wTh) -- (-1.5,0) -- (-\sL/16,0) -- (-\sL/16,-\wTh)
         -- (\sL/16,-\wTh) -- (\sL/16,0) -- (3*\sL/16,0) -- (3*\sL/16,-\wTh)
         -- (5*\sL/16,-\wTh) -- (5*\sL/16,0) -- (\sL/2,0);
      \end{scope}
      \begin{scope}[shift={(2,{2*sqrt(3)})},rotate=-60]
        \draw[base] (-\sL/2,0) -- (\sL/2,0);
      \draw[main] (-\sL/2,0) -- (-5*\sL/16,0) -- (-5*\sL/16,-\wTh)
         -- (-3*\sL/16,-\wTh) -- (-1.5,0) -- (-\sL/16,0) -- (-\sL/16,-\wTh)
         -- (\sL/16,-\wTh) -- (\sL/16,0) -- (3*\sL/16,0) -- (3*\sL/16,-\wTh)
         -- (5*\sL/16,-\wTh) -- (5*\sL/16,0) -- (\sL/2,0);
      \end{scope}
    \end{tikzpicture}
    \caption{Top (plexi glass) plate form}
  \end{figure}
  \tikzexternaldisable
  \begin{figure}\centering
    \tikzsetnextfilename{skecth-side}
    \begin{tikzpicture}
      \draw[base] (-\sL/2,\h) rectangle (\sL/2,0);
      \draw[main] ({-(\sL/2-\wTh*cot(60))},\h) -- (-5*\sL/16,\h)
        -- (-5*\sL/16,\h+\pTh) -- (-3*\sL/16,8+\pTh)
        -- (-3*\sL/16,\h) -- (-\sL/16,\h) -- (-\sL/16,\h+\pTh)
        -- (\sL/16,\h+\pTh) -- (\sL/16,\h) -- (3*\sL/16,\h)
        -- (3*\sL/16,\h+\pTh) -- (5*\sL/16,\h+\pTh) -- (5*\sL/16,\h)
        -- ({\sL/2- \wTh*cot(60)-\wTh},\h);
        %start top left - main frame
    %   \draw[main] ({-(4-\wTh*cot(60))},8) -- ({-(4-\wTh*cot(60))},7) --
    %         ({-(4-\wTh*cot(60))+\wTh},7) -- ({-(4-\wTh*cot(60))+\wTh},6) --
    %         ({-(4-\wTh*cot(60))},6) -- ({-(4-\wTh*cot(60))},5) --
    %         ({-(4-\wTh*cot(60))+\wTh},5) -- ({-(4-\wTh*cot(60))+\wTh},4) --
    %         ({-(4-\wTh*cot(60))},4) -- ({-(4-\wTh*cot(60))},3) --
    %         ({-(4-\wTh*cot(60))+\wTh},3) -- ({-(4-\wTh*cot(60))+\wTh},2) --
    %         ({-(4-\wTh*cot(60))},2) -- ({-(4-\wTh*cot(60))},1) --
    %         ({-(4-\wTh*cot(60))+\wTh},1) -- ({-(4-\wTh*cot(60))+\wTh},0) --%
    %         % bottom line
    %         ({(4-\wTh*cot(60))},0) --
    %         % and upwards
    %         ({(4-\wTh*cot(60))},1) -- ({(4-\wTh*cot(60))-\wTh},1) --%
    %         ({(4-\wTh*cot(60))-\wTh},2) -- ({(4-\wTh*cot(60))},2) --
    %         ({(4-\wTh*cot(60))},3) -- ({(4-\wTh*cot(60))-\wTh},3) --%
    %         ({(4-\wTh*cot(60))-\wTh},4) -- ({(4-\wTh*cot(60))},4) --
    %         ({(4-\wTh*cot(60))},5) -- ({(4-\wTh*cot(60))-\wTh},5) --%
    %         ({(4-\wTh*cot(60))-\wTh},6) -- ({(4-\wTh*cot(60))},6) --
    %         ({(4-\wTh*cot(60))},7) -- ({(4-\wTh*cot(60))-\wTh},7) --%
    %         ({(4-\wTh*cot(60))-\wTh},8);
    % % bottom plate
    % \draw [magnets] (-2-\mTh/2,4-\mTh/2) rectangle (-2+\mTh/2,4+\mTh/2);
    % \draw [magnets] (2-\mTh/2,4-\mTh/2) rectangle (2+\mTh/2,4+\mTh/2);
    % \node (A) at (2,4) {\large N};
    % \node (A) at (-2,4) {\large S};
    % % 4944
    % % bottom plate holes
    % \draw [holes] (-2.5,2-\wTh/2) rectangle (-1.5,2+\wTh/2);
    % \draw [holes] (-.5,2-\wTh/2) rectangle (.5,2+\wTh/2);
    % \draw [holes] (1.5,2-\wTh/2) rectangle (2.5,2+\wTh/2);
    % %
    % % Measures
    % %
    % % bottom
    %    \draw[measure] ({- ( 4- \wTh*cot(60) )},-.15) -- (-2.5,-.15)
    %      node[midway,below] {\pgfmathparse{10*(1.5 - \wTh*cot(60))}%
    %      \pgfmathprintnumber{\pgfmathresult}\,mm};
    %     \draw[measure] (-2.5,-.15-.25) -- (-1.5,-.15-.25) node[midway, below]{\(10\)\,mm};
    %     \draw[measure] (-1.5,-.15) -- (-.5,-.15) node [midway, below]{\(10\)\,mm};
    %     \draw[measure] (-.5,-.15-.25) -- (.5,-.15-.25) node [midway, below]{\(10\)\,mm};
    %     \draw[measure] (.5,-.15) -- (1.5,-.15) node [midway, below]{\(10\)\,mm};
    %     \draw[measure] (1.5,-.15-.25) -- (2.5,-.15-.25) node [midway, below] {\(10\)\,mm};
    %      \draw[measure] ({( 4- \wTh*cot(60) )},-.15) -- (2.5,-.15)
    %      node[midway,below] {\pgfmathparse{10*(1.5 - \wTh*cot(60))}%
    %      \pgfmathprintnumber{\pgfmathresult}\,mm};
    %   \draw[measure] ({- ( 4- \wTh*cot(60) )},-.8) -- ({4- \wTh*cot(60)},-.8)
    %     node[midway,below] {\pgfmathparse{10*(8 - 2*\wTh*cot(60))}%
    %     \pgfmathprintnumber{\pgfmathresult}\,mm};
    %   \draw[measure] (-4,-1.45) -- (4,-1.45) node [midway, below] {\(80\)\,mm};
    % % left
    % \draw[measure] (4.15+.25,8+\wTh) -- (4.15+.25,8) node[midway,right] {\pgfmathparse{10*\wTh}\pgfmathprintnumber{\pgfmathresult}\,mm};
    % \draw[measure] (4.15,8) -- (4.15,7) node[midway,right] {\(10\)\,mm};
    % \draw[measure] (4.15+.25,7) -- (4.15+.25,6) node[midway,right] {\(10\)\,mm};
    % \draw[measure] (4.15,6) -- (4.15,5) node[midway,right] {\(10\)\,mm};
    % \draw[measure] (4.15+.25,5) -- (4.15+.25,4) node[midway,right] {\(10\)\,mm};
    % \draw[measure] (4.15,4) -- (4.15,3) node[midway,right] {\(10\)\,mm};
    % \draw[measure] (4.15+.25,3) -- (4.15+.25,2) node[midway,right] {\(10\)\,mm};
    % \draw[measure] (4.15,2) -- (4.15,1) node[midway,right] {\(10\)\,mm};
    % \draw[measure] (4.15+.25,1) -- (4.15+.25,0) node[midway,right] {\(10\)\,mm};
    \end{tikzpicture}
  \end{figure}
\end{document}